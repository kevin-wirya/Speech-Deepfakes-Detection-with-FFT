\documentclass[conference]{IEEEtran}
\IEEEoverridecommandlockouts

\usepackage{cite}
\usepackage{amsmath,amssymb,amsfonts}
\usepackage{algorithmic}
\usepackage{graphicx}
\usepackage{textcomp}
\usepackage{xcolor}
\usepackage{tikz}
\usepackage{algorithm}
\usepackage{algorithmic}
\usetikzlibrary{positioning}
\usepackage[
    colorlinks=true,
    urlcolor=blue
]{hyperref}

\newcommand{\blueulinehref}[2]{\href{#1}{\textcolor{blue}{\underline{\textit{#2}}}}}

\def\BibTeX{{\rm B\kern-.05em{\sc i\kern-.025em b}\kern-.08em
    T\kern-.1667em\lower.7ex\hbox{E}\kern-.125emX}}

\tikzset{
    mynode/.style={circle, draw, fill=white, inner sep=2pt, font=\small},
    mylabel/.style={font=\small, align=center},
    myarrow/.style={-stealth, very thick}
}

\begin{document}

\title{Speech Deepfakes Detection with Fast Fourier Transform using Complex Linear Algebra}

\author{\IEEEauthorblockN{Kevin Wirya Valerian - 13524019}
\IEEEauthorblockA{\textit{Program Studi Teknik Informatika} \\
\textit{Sekolah Teknik Elektro dan Informatika}\\
Institut Teknologi Bandung, Jl. Ganesha 10 Bandung 40132, Indonesia \\
\blueulinehref{kevin.wirya.valerian@gmail.com}{kevin.wirya.valerian@gmail.com}
\blueulinehref{13524019@std.stei.itb.ac.od}{13524019@std.stei.itb.ac.id}}
}

\maketitle

\begin{abstract}
Nowadays, the growing realism of speech deepfakes demands detection methods grounded in fundamental signal theory rather than purely data-driven models. A well-known speech deepfake detection approach is based on the Fast Fourier Transform (FFT), rooted in complex linear algebra and geometric signal representations. Speech signals are modeled as vectors in complex vector spaces, where the FFT acts as a structured linear transformation projecting time-domain data onto orthonormal bases of complex exponentials. From a geometric perspective, genuine and synthetic speech occupy distinct regions in high-dimensional spectral spaces. Subtle inconsistencies in phase alignment, spectral symmetry, and energy distribution are captured through algebraic and geometric measures, enabling robust and interpretable deepfake detection. The study contributes significantly to rapid AI and real speech distinction.
\end{abstract}

\begin{IEEEkeywords}
Fast fourier transform, deepfakes detection, complex linear algebra, spectral geometry
\end{IEEEkeywords}

% ===================================
% Import chapter files
% ===================================
\section{Introduction}

Recent advances in speech synthesis and voice conversion have enabled machines to generate highly realistic human speech. Modern text-to-speech and voice cloning systems are capable of producing audio that is increasingly difficult to distinguish from genuine human recordings. A study reports that human participants were only able to accurately distinguish real from AI-generated voices with an accuracy of 
70.4\%. This leads to a serious risks that can be posed by speech deepfakes in areas such as biometric authentication and digital forensics. Reports indicate that voice-based fraud has increased significantly in recent years, with financial losses reaching billions of dollars globally. Deepfake-related losses have already reached \$1.56 billion, with over \$1 billion occurring in 2025 alone.

While deep learning models have achieved impressive performance in generating natural-sounding speech, detecting such synthetic audio remains a challenging task. Many existing detection methods rely heavily on large neural networks trained on specific datasets. Although effective, these approaches often lack interpretability and tend to degrade when exposed to unseen deepfake generation methods. Moreover, purely data-driven models can be computationally expensive and difficult to analyze, making them less suitable and reliable at present time.

Speech signals, however, are fundamentally mathematical objects. A digital audio signal can be represented as a finite sequence of samples, which naturally forms a vector in a high-dimensional space. Transforming this signal into the frequency domain using the Fast Fourier Transform (FFT) reveals its spectral structure, including energy distribution and phase behavior. These properties are governed by well-established principles of linear algebra and geometry, such as vector spaces, orthonormal bases, inner products, and unitary transformations. Importantly, while deepfake speech models often reproduce spectral magnitudes accurately, subtle geometric inconsistencies in phase relationships and spectral structure may persist.

Motivated by this observation, this paper proposes a speech deepfake detection approach grounded in the linear algebraic and geometric foundations of the FFT. The proposed method aims to distinguish genuine and synthetic speech through interpretable mathematical measures. The development of this approach is supported by core theories from complex linear algebra and geometric signal analysis, which together provide a principled framework for understanding and detecting anomalies in synthetic speech. hello hi

\section{Theoretical Framework}

\subsection{Complex Numbers and Geometric Representation}

A complex number is defined as
\begin{equation}
z = a + jb, \quad a, b \in \mathbb{R}
\end{equation}
where $a$ is the real part and $b$ is the imaginary part, with $j = \sqrt{-1}$. Both $i$ and $j$ represent the same imaginary unit. We use $j$ since it is commonly used for signal processing.

Complex numbers admit a geometric interpretation in the complex plane, where the real part corresponds to the horizontal axis and the imaginary part to the vertical axis. This representation allows us to visualize complex numbers as vectors emanating from the origin.

Every non-zero complex number can be expressed in polar form
\begin{equation}
z = r e^{j\theta}
\end{equation}
where
\begin{itemize}
    \item $r = |z| = \sqrt{a^2 + b^2}$ is the \textbf{magnitude}, representing the distance from the origin
    \item $\theta = \arg(z) = \arctan(b/a)$ is the \textbf{phase}, representing the angle from the positive real (horizontal) axis
\end{itemize}

\begin{figure}[htbp]
\centerline{\includegraphics[width=6cm]{img/complex_plane.png}}
\caption{Argand Plane (Complex Plane)}
\small \centering (Source: \href{https://helpingwithmath.com/complex-plane/}{https://helpingwithmath.com/complex-plane/})
\label{fig}
\end{figure}

The exponential form relates to trigonometry via Euler's formula.
\begin{equation}
e^{j\theta} = \cos\theta + j\sin\theta
\end{equation}

The FFT spectrum of audio signals consists of complex-valued coefficients. Each coefficient $X_k$ can be decomposed into magnitude and phase components. The magnitude $|X_k|$ represents the energy at frequency bin $k$, while the phase $\angle X_k$ encodes temporal structure. A frequency bin is a specific range on the frequency axis used to group and analyze data. Our detection method exploits the observation that human speech exhibits \textbf{regular} phase structure, whereas AI-generated speech tends to have \textbf{irregular} phase structure.

\subsection{Complex Vector Spaces $\mathbb{C}^N$}

The set $\mathbb{C}^N$ of all $N$-tuples of complex numbers forms a vector space over the field $\mathbb{C}$. A vector $\mathbf{x} \in \mathbb{C}^N$ is written as
\begin{equation}
\mathbf{x} = \begin{pmatrix} x_0 \\ x_1 \\ \vdots \\ x_{N-1} \end{pmatrix}, \quad x_n \in \mathbb{C}
\end{equation}

Digital audio signals are naturally elements of $\mathbb{R}^N$ in the time domain and $\mathbb{C}^N$ in the frequency domain after FFT transformation. The standard inner product on $\mathbb{C}^N$ is defined as
\begin{equation}
\langle \mathbf{x}, \mathbf{y} \rangle = \sum_{n=0}^{N-1} x_n \overline{y_n}
\end{equation}
where $\overline{y_n}$ denotes the complex conjugate of $y_n$. This inner product induces the Euclidean norm.
\begin{equation}
\|\mathbf{x}\| = \sqrt{\langle \mathbf{x}, \mathbf{x} \rangle} = \sqrt{\sum_{n=0}^{N-1} |x_n|^2}
\end{equation}

The squared norm $\|\mathbf{x}\|^2$ represents the total energy of the signal.

\subsection{Discrete and Fast Fourier Transform}

The Discrete Fourier Transform (DFT) converts a time-domain signal $\mathbf{x} \in \mathbb{C}^N$ to its frequency-domain representation $\mathbf{X} \in \mathbb{C}^N$ via:
\begin{equation}
X_k = \sum_{n=0}^{N-1} x_n e^{-j2\pi kn/N}, \quad k = 0, 1, \ldots, N-1
\end{equation}

This transformation can be interpreted as computing the inner product of the signal with complex exponential basis vectors at each frequency $k$. The inverse DFT reconstructs the time-domain signal:
\begin{equation}
x_n = \frac{1}{N}\sum_{k=0}^{N-1} X_k e^{j2\pi kn/N}, \quad n = 0, 1, \ldots, N-1
\end{equation}

The naive DFT requires $O(N^2)$ complex multiplications. The Fast Fourier Transform (FFT) reduces this to $O(N \log N)$ using a divide-and-conquer strategy. The key insight is that DFT can be decomposed into smaller DFTs of even and odd indexed elements.

Let $\omega_N = e^{-j2\pi/N}$ be the primitive $N$-th root of unity. For $N = 2^m$, we split the DFT:
\begin{align}
X_k &= \sum_{n=0}^{N-1} x_n \omega_N^{kn} \\
&= \sum_{n=0}^{N/2-1} x_{2n} \omega_N^{k(2n)} + \sum_{n=0}^{N/2-1} x_{2n+1} \omega_N^{k(2n+1)} \\
&= \sum_{n=0}^{N/2-1} x_{2n} (\omega_N^2)^{kn} + \omega_N^k \sum_{n=0}^{N/2-1} x_{2n+1} (\omega_N^2)^{kn}
\end{align}

Since $\omega_N^2 = \omega_{N/2}$, this becomes
\begin{equation}
X_k = E_k + \omega_N^k O_k
\end{equation}
where $E_k$ is the DFT of even-indexed elements and $O_k$ is the DFT of odd-indexed elements, both of size $N/2$.

Using the symmetry property $X_{k+N/2} = E_k - \omega_N^k O_k$, we compute both halves with a single recursion:
\begin{align}
X_k &= E_k + \omega_N^k O_k, \quad k = 0, \ldots, N/2-1 \\
X_{k+N/2} &= E_k - \omega_N^k O_k, \quad k = 0, \ldots, N/2-1
\end{align}

The recursion bottoms out at $N=1$, where $X_0 = x_0$. The total complexity satisfies:
\begin{equation}
T(N) = 2T(N/2) + O(N) = O(N \log N)
\end{equation}

\begin{figure}[htbp]
\centerline{\includegraphics[width=5cm]{img/time_to_freq_fft}}
\caption{Time to Frequency Mapping By FFT (Complex Plane)}
\small \centering (Source: \href{https://www.sciencedirect.com/topics/engineering/fast-fourier-transform}{https://www.sciencedirect.com/topics/engineering/fast-fourier-transform})
\label{fig}
\end{figure}

The FFT preserves several important properties crucial for our detection method:

\text{1. Energy Preservation (Parseval's Theorem)}
\begin{equation}
\sum_{n=0}^{N-1} |x_n|^2 = \frac{1}{N}\sum_{k=0}^{N-1} |X_k|^2
\end{equation}

This ensures that total signal energy is conserved between time and frequency domains.

\text{2. Linearity} 

FFT$(\alpha \mathbf{x} + \beta \mathbf{y}) = \alpha$FFT$(\mathbf{x}) + \beta$FFT$(\mathbf{y})$, allowing us to analyze signal components independently.

\text{3. Symmetry for Real Signals:} 

When $x_n \in \mathbb{R}$, we have $X_{N-k} = \overline{X_k}$, so only the first $N/2$ coefficients need to be stored.

For our implementation, audio signals are real-valued in the time domain, so we extract magnitude and phase from the positive frequency components $k = 0, \ldots, N/2-1$.

\subsection{Magnitude and Phase as Complex Algebraic Objects}

Each FFT coefficient admits the polar decomposition:
\begin{equation}
X_k = |X_k| e^{j\theta_k}
\end{equation}
where:
\begin{itemize}
    \item $|X_k| = \sqrt{\text{Re}(X_k)^2 + \text{Im}(X_k)^2}$ is the spectral magnitude
    \item $\theta_k = \arctan\left(\frac{\text{Im}(X_k)}{\text{Re}(X_k)}\right)$ is the spectral phase
\end{itemize}

Phase coherence measures the consistency of phase relationships across frequency bins. For a phase spectrum $\boldsymbol{\theta} = (\theta_0, \theta_1, \ldots, \theta_{N-1})$, we define
\begin{equation}
C(\boldsymbol{\theta}) = \left| \frac{1}{N} \sum_{k=0}^{N-1} e^{j\theta_k} \right|
\end{equation}

This metric ranges from 0 to 1:
\begin{itemize}
    \item $C = 1$: Perfectly coherent
    \item $C = 0$: Completely random
    \item $0 < C < 1$: Partial coherence
\end{itemize}

The phase velocity (or phase derivative) measures the smoothness of phase progression
\begin{equation}
v_k = \theta_{k+1} - \theta_k \pmod{2\pi}
\end{equation}

The variance of phase velocity indicates regularity:
\begin{equation}
\sigma_v^2 = \text{Var}(v_k)
\end{equation}

Low variance suggests smooth and natural phase evolution, while high variance indicates abrupt phase changes typical of synthesis artifacts.

In our implementation, we quantify phase smoothness using the mean absolute phase gradient:
\begin{equation}
\bar{v} = \frac{1}{N-1}\sum_{k=0}^{N-2} |v_k|
\end{equation}
where smaller values of $\bar{v}$ indicate smoother phase transitions characteristic of natural speech, while larger values suggest the discontinuous phase patterns often present in synthesized audio.

Modern AI voice synthesis (neural vocoders, GANs) primarily optimizes for perceptually accurate magnitude spectra because human hearing is more sensitive to magnitude than phase. However, these models often fail to produce coherent phase structure because phase reconstruction is mathematically ill-conditioned. 

\subsection{Spectral Entropy and Energy Distribution}

Spectral entropy quantifies the concentration of energy across the frequency spectrum. For a magnitude spectrum $\mathbf{M} = (|X_0|, |X_1|, \ldots, |X_{N-1}|)$, we first normalize to obtain a probability distribution
\begin{equation}
p_k = \frac{|X_k|}{\sum_{i=0}^{N-1} |X_i|}
\end{equation}

The spectral entropy is then defined as
\begin{equation}
H(\mathbf{M}) = -\sum_{k=0}^{N-1} p_k \log p_k
\end{equation}

This metric characterizes the distribution of spectral energy
\begin{itemize}
    \item \textbf{Low entropy}: Energy concentrated in few frequency bins (typical of human speech)
    \item \textbf{High entropy}: Energy spread uniformly across frequencies (may indicate synthesis artifacts)
\end{itemize}

The spectral L2 norm
\begin{equation}
\|\mathbf{M}\|_2 = \sqrt{\sum_{k=0}^{N-1} |X_k|^2}
\end{equation}
represents the total energy in the signal and serves as a normalization factor for geometric distance computations.

\subsection{Window-Based Phase Coherence Analysis}

While global phase coherence provides an overall measure, local phase patterns can reveal subtle artifacts. We employ a sliding window approach to compute localized coherence. For a window size $w$, the local phase coherence at position $i$ is
\begin{equation}
C_i^{(w)} = \frac{1}{w}\left|\sum_{k=i}^{i+w-1} e^{j\theta_k}\right|
\end{equation}

The overall phase coherence is then the mean of all window coherences:
\begin{equation}
C_{\text{window}} = \frac{1}{N-w}\sum_{i=0}^{N-w-1} C_i^{(w)}
\end{equation}

This windowed approach offers several advantages:
\begin{itemize}
    \item Captures local phase consistency patterns
    \item More robust to isolated phase discontinuities
    \item Reveals frequency-dependent phase artifacts common in AI synthesis
\end{itemize}

The window size $w$ is typically chosen based on the expected correlation length of natural speech phase patterns.

\subsection{Data Clustering}

In our framework, audio samples form clusters in the complex feature space. Let $\mathcal{H} = \{\mathbf{h}_1, \ldots, \mathbf{h}_M\}$ be the set of feature vectors from human speech samples, with centroid
\begin{equation}
\boldsymbol{\mu}_H = \frac{1}{M}\sum_{i=1}^{M} \mathbf{h}_i
\end{equation}

Similarly, AI-generated samples form a cluster $\mathcal{A}$ with centroid $\boldsymbol{\mu}_A$. The optimal decision boundary lies at the midpoint between centroids:
\begin{equation}
\tau = \frac{\boldsymbol{\mu}_H + \boldsymbol{\mu}_A}{2}
\end{equation}

For a test sample with phase coherence $C$, the classification rule is:
\begin{equation}
\text{decision}(C) = \begin{cases}
\text{HUMAN} & \text{if } C > \tau \\
\text{AI-GENERATED} & \text{if } C \leq \tau
\end{cases}
\end{equation}
where $\tau$ is the optimal threshold computed from the training data statistics.

\subsection{Mahalanobis-Like Distance and Confidence Estimation}

To quantify classification confidence, we compute standardized distances from the test sample to each cluster centroid. Given the phase coherence $C$ of a test sample, and the statistics $(\mu_H, \sigma_H)$ and $(\mu_A, \sigma_A)$ from the human and AI training sets respectively, we define:

\begin{align}
d_H(C) &= \frac{|C - \mu_H|}{\sigma_H + \epsilon} \\
d_A(C) &= \frac{|C - \mu_A|}{\sigma_A + \epsilon}
\end{align}

where $\epsilon$ is a small constant to prevent division by zero. These distances are analogous to Mahalanobis distances in one dimension, accounting for the variance of each class.

The confidence score is computed from the relative distances:
\begin{equation}
\text{confidence} = 1 - \frac{\min(d_H, d_A)}{\max(d_H, d_A) + \epsilon}
\end{equation}

This confidence metric has the following properties:
\begin{itemize}
    \item $\text{confidence} \in [0, 1]$
    \item High confidence ($\approx 1$): Test sample is much closer to one cluster than the other
    \item Low confidence ($\approx 0$): Test sample is equidistant from both clusters (ambiguous case)
\end{itemize}

The predicted class is the one with smaller distance:
\begin{equation}
\text{prediction} = \begin{cases}
\text{HUMAN} & \text{if } d_H < d_A \\
\text{AI-GENERATED} & \text{otherwise}
\end{cases}
\end{equation}

This geometric approach provides both a classification decision and a measure of certainty, essential for practical deployment where uncertain predictions may require human review.


\section{Methodology}

The detection system operates through a four-stage pipeline. Those are audio preprocessing, frequency domain transformation, feature extraction, and geometric distance-based classification. The core hypothesis is that human speech exhibits distinct phase relationships in the frequency domain that differ systematically from AI-generated audio. These differences arise from fundamental mechanisms that human speech originates from physical vocal cord vibrations and acoustic resonance, while AI-generated speech is synthesized digitally, potentially introducing artifacts in phase structure.

\subsection{Signal Processing}

The audio signal is first loaded and normalized to a consistent sampling rate and amplitude range. Stereo recordings are converted to mono by averaging channels. The preprocessed time-domain signal then undergoes Fast Fourier Transform (FFT), converting it from amplitude-over-time representation to amplitude-and-phase-over-frequency representation.

The FFT produces a complex-valued vector where each element corresponds to a frequency component. The magnitude represents energy at that frequency, while the phase angle captures timing relationships. The system retains only positive frequency components, exploiting the symmetry property of real-valued signals. This frequency domain representation becomes the foundation for all subsequent analysis.

\subsection{Feature Extraction}

From the FFT output, the system extracts three key features characterizing the signal's geometric properties. First, phase coherence serves as the primary discriminative feature. They are used for quantify how consistently phase transitions across adjacent frequency bins. The computation analyzes phase values in sliding windows across the spectrum. High coherence indicates regular phase transitions typical of human speech. On the other hand, low coherence suggests random or artificial patterns common in nonhuman speech.

Second, phase velocity measures the rate of phase change across frequency. Human speech typically exhibits smoother phase curves due to continuous physical sound production, while AI-generated speech may show abrupt transitions.

Third, spectral entropy quantifies energy distribution across frequencies. Low entropy indicates concentration in specific bands, while high entropy suggests uniform distribution.

\subsection{Classification Strategy}

The classification employs a geometric distance-based approach rather than machine learning. Before classification, the system computes reference statistics from known human and AI-generated speech samples. For each dataset, phase coherence values are extracted from multiple files, and statistical measures (mean and standard deviation) are computed.

For a test audio sample, the system extracts its phase coherence and computes normalized geometric distances to both human and AI-generated reference distributions. The distance to each class measures how many standard deviations the test sample deviates from that class's mean, similar to Mahalanobis distance. The system classifies the sample as belonging to the class with the smaller geometric distance. Confidence is quantified by the ratio of distances, reflecting how much closer the sample is to its predicted class compared to the alternative class.

This geometric distance-based strategy offers immediate interpretability through clear distance metrics, minimal calibration requirements, and avoidance of overfitting risks. The approach prioritizes mathematical transparency and geometric reasoning over complex learned parameters, making the classification process fully explainable through linear algebra principles.

\section{Implementation}

\subsection{Signal Processing}
\begin{figure}[htbp]
\centerline{\includegraphics[width=8cm]{img/code_load_wav.png}}
\caption{Loading Audio Files}
\small \centering (Source: Author)
\label{fig}
\end{figure}
First, audio file input was handled by first detecting the file extension. For WAV files, it uses scipy.io.wavfile.read to load the audio data and metadata, while MP3 files are loaded using the librosa library. The method then applies preprocessing steps. The signal is normalized to a floating-point range of [-1, 1]. 

\begin{figure}[htbp]
\centerline{\includegraphics[width=8cm]{img/code_compute_features.png}}
\caption{Computing Mathematical Features of Audio}
\small \centering (Source: Author)
\label{fig}
\end{figure}
There is a part of the code to transform the time-domain signal into the frequency domain using Fast Fourier Transform (FFT) via scipy.fftpack.fft. All results—complex coefficients, magnitude, phase, and frequency—are organized into a dictionary for downstream feature calculations.

\begin{figure}[htbp]
\centerline{\includegraphics[width=8cm]{img/code_compute_after_features.png}}
\caption{Computing Phase Coherence, Phase Velocity, and Spectral Entropy}
\small \centering (Source: Author)
\label{fig}
\end{figure}
Spectral coherence is computed using a sliding window approach. The method returns both the overall coherence as the mean of all window coherences and an array of per-window values for diagnostic purposes. Next, the the smoothness of phase is quantified by computing the firt-order difference between adjacent phase. Last, there exists a part for extracting L2 norm, then normalize the magnitude vector using it. By using entropy formula, in the end, three metrics are returned in a dictionary. Extraction of all features from a file need to be executed for a complete information gathering.

\begin{figure}[htbp]
\centerline{\includegraphics[width=8cm]{img/code_extract.png}}
\caption{Extraction}
\small \centering (Source: Author)
\label{fig}
\end{figure}

\newpage
\subsection{Reference Statistics}
\begin{figure}[htbp]
\centerline{\includegraphics[width=8cm]{img/code_get_wav.png}}
\caption{Obtaining WAV Files}
\small \centering (Source: Author)
\label{fig}
\end{figure}
All audio files inside a certain directory are recursively searched by using glob patterns to find files matching *.wav and *.mp3. The method returns a sorted list of Path objects representing all discovered audio files.

\begin{figure}[htbp]
\centerline{\includegraphics[width=8cm]{img/code_compute_statistics.png}}
\caption{Computing Statistics}
\small \centering (Source: Author)
\label{fig}
\end{figure}
Baseline statistics must be computed to process human and nonhuman datasets. After processing all files, the method computes aggregate statistics for features such as mean, standard deviation, minimum, and maximum values. Then, the decision threshold is calculated.

\begin{figure}[htbp]
\centerline{\includegraphics[width=8cm]{img/code_save_load.png}}
\caption{Computing and Saving Reference Statistics}
\small \centering (Source: Author)
\label{fig}
\end{figure}
The rest of the code is used to save and load the statistics. When the application is first started, the data will get computed by the functions there.

\subsection{Detector}
\begin{figure}[htbp]
\centerline{\includegraphics[width=8cm]{img/code_geo_distance.png}}
\caption{Computing Geometric Distance}
\small \centering (Source: Author)
\label{fig}
\end{figure}
Geometric distance calculation is used for confidence classification. It is derived from the formula mentioned in the theoretical framework. It handles errors and also division by zero.

\begin{figure}[htbp]
\centerline{\includegraphics[width=8cm]{img/code_predict.png}}
\caption{Predicting results}
\small \centering (Source: Author)
\label{fig}
\end{figure}
Prediction performs the main classification task.  The method applies pre-computed threshold to classify whether the audio is human or nonhuman. It calculates the confidence score via this data. We used the help of \text{predict\_batch} to predict multiple audio files. 

\subsection{Application}
\begin{figure}[htbp]
\centerline{\includegraphics[width=8cm]{img/code_app.png}}
\caption{Predicting results}
\small \centering (Source: Author)
\label{fig}
\end{figure}
The Flask web application serves as the user interface for the deepfake detection system. When initializing detector, system automatically checks for the existence of reference statistics and computes them from the training datasets if unavailable, ensuring the system operates well on first launch. Upon receiving a file, the endpoint validates the file format, saves it temporarily to disk, invokes the deepfake detector to perform classification, and returns a JSON response containing the prediction label (human or AI-generated), confidence score, and detailed spectral metrics including phase coherence, geometric distances to both classes, phase velocity, spectral entropy, and L2 norm. Additional endpoints provide system status checks and access to reference statistics for diagnostic purposes.
\section{Case Analysis}

% Write your discussion here


\section{Conclusion}

In conclusion, this paper presents a novel approach to AI-generated audio detection by leveraging phase-based analysis through the fast fourier transform (FFT). The system created by the author employs three key metrics—phase coherence, phase velocity, and spectral entropy to distinguish between human and nonhuman speech. The mathematical foundation utilizes windowing techniques to segment audio signals, applies FFT for frequency domain transformation, and extracts phase information to identify subtle artifacts in synthetic speech. These metrics are integrated through a weighted classification approach that evaluates the temporal consistency, spectral distribution, and phase relationships inherent in audio signals, providing a computationally efficient framework for deepfake detection.

However, the system exhibits certain limitations that affect its detection accuracy regarding the stability and similarity parameter since it shows no major findings. Needless to say, this system will be obsolete. The system might fail to distinguish authentic audio from synthetic speech when environmental factors such as background noise, low-quality recordings, or post-processing effects are present, as these conditions can introduce phase distortions similar. 

Several improvements are recommended to increase system's performance. First, incorporating additional spectral features such as mel-frequency cepstral coefficients (MFCCs) or formant analysis could provide complementary information to phase-based metrics. Second, implementing machine learning classifiers trained on diverse datasets would enable adaptive threshold determination and better generalization across different AI synthesis models. Third, developing noise-robust preprocessing techniques would improve performance in real-world scenarios with varying audio quality. Last but not least, there are many approaches for detecting speech deepfakes, using FFT is not always the most optimal approach. Nevertheless, as long as technology advancement is concerned, FFT is one of the most efficient algorithms to detect AI-generated speech in an optimized time.

% ===================================
% Bibliography
% ===================================
\newpage
\begin{thebibliography}{00}
\bibitem{b1} Ni, Y., et al. (2024). "A Deepfake Detection Algorithm Based on Fourier Transform of Biological Signal." Tech Science Press. (Accessed on December 16, 2025) \href{https://www.techscience.com/cmc/v79n3/57116}{https://www.techscience.com/cmc/v79n3/57116}
\bibitem{b2} Mai, K. T., Bray, S., Davies, T., and Griffin, L. D.Warning: Humans cannot reliably detect speech deepfakes.PLOS ONE, 18(8):1–20, 2023. (Accessed on December 19, 2025) \href{https://journals.plos.org/plosone/article?id=10.1371/journal.pone.0285333}{https://journals.plos.org/plosone/article?id=10.1371/journal.pone.0285333}
\end{thebibliography}

\section*{Statement}
Hereby I declare that this paper that I have written is my own work, not a reproduction or translation of someone else's work and not plagiarized.

\begin{flushright}
Bandung, 20 Juni 2025\\
\begin{minipage}[t]{0.3\textwidth} 
    \centering
    \hspace{2.2cm}
    \includegraphics[width=3cm]{img/sign.jpg} 
\end{minipage} \\
Kevin Wirya Valerian\\
13524019
\end{flushright}

\end{document}
